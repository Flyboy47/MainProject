\documentclass[a4paper]{article}

\usepackage{graphicx}
\usepackage{float}


\title{\Huge{Tender Response} \\[1cm] \LARGE{to Facial Recognition System for Communities \\[3cm]}}

\author{BiTwise Developers \\[0.5cm]}

\date{18 March, 2014}
 
\begin{document}

\maketitle
\newpage

\tableofcontents
\newpage

	\section{General}
	
		\subsection{Purpose of Document}
		
		This document provides a description of the proposed solution submitted by Bitwise Developers in response to the invitation 
		for bids on tender for the Facial recognition system for communities from Quant Solutions.
		
		\subsection{Introduction}
		
		A community watch has identified a need to monitor and identify individuals who live and operate in their comminity, in order to 
		distinguish between people who belong in the area, and those who might be up to no good. 
		A possible solution to the problem has been identified by Quant Solutions, who envision a computer system that can actively monitor
		and identify individuals who operate in the area by using IP camera surveillance, and a back-end database to process and store captured
		images. The processed data can then be queried by a member of the community watch or police, who is authorised to and tasked with interrogating
		suspicious individuals. This can be done either through a mobile device, or a web interface.
		
	\section{Executive Summary}

		\subsection{Stakeholders}
		
			\subsubsection{Project Owner}
			
				\begin{description}
					\item[Client:] Quant Solutions (pty) ltd
					\item[Contact Person:] Kobus Wolvaardt
					\item[E-mail:] kobuswolf@gmail.com
				\end{description}
				
			\subsubsection{BiTwise Developers}
				
				BiTwise Developers are a unique collection of aspiring IT professionals, each with ambition and sound judgement. We are all in our final year of study at the University of Pretoria.
				We strive to accomplish each task effectively and efficiently, in such a way that it always meets our high standards. We are responsible and reliable. Each individual brings something 
				unique to our team. We all get along with each other and work well as a unit.
				
				\begin{description}
					\item[Lead Developer:] Jacques Lewis
					\item[E-mail:] u28183488@tuks.co.za
				\end{description}
				
				\begin{description}
					\item[Developer:] Priscilla Hammond
					\item[E-mail:] u11025477@tuks.co.za
				\end{description}
				
				\begin{description}
					\item[Developer:] Francois Oberholzer
					\item[E-mail:] u12039803@tuks.co.za
				\end{description}
			
	\section{Technical Proposal}
	
		\subsection{Project Description}
		
			\subsubsection{Problem Statement}
			
			A number of real time video streams, each with possibly different resolutions or orientations, need to be monitored for
			faces of individuals who operate in and around a neighbourhood. These faces need to be analysed and given individual identities,
			such that a database can be established and maintained and later used to identify specific individuals.
			
			\subsubsection{Objectives of the project}
			
				\begin{itemize}
				
					\item Gather facial feature data to be used to identify individual people
					
					\item Classify each individual
					
					\item Use a client application to attempt to identify a person of interest against the database of individuals
			
				\end{itemize}
			
		\subsection{Proposed Solution}
		
		We believe that we can solve the problem by using our knowledge and resources, and by applying our skills as software developers.
		We propose a possible solution as follows:
		
			\subsubsection{Solution Overview}
			
			Our solution is to employ OpenCV libraries to do most of the facial detection and recognition tasks. These include, but are not limited to:
				
				\begin{itemize}
				
					\item Detecting a face in a video frame
					
					\item Cropping an image to remove superfluous pixels
					
					\item Pre-processing the cropped image
					
					\item Creating Eigenfaces for both the dataset of known faces, as well as for comparing input faces to known faces
				
				\end{itemize}
			Our solution includes an interface that allows an authorised member of the community watch to attempt to identify an individual by providing 
			an image of that individual, which will then be compared to the set of known faces in the database. We also privide a management component,
			from where the system can be managed and monitored.
			
			\subsubsection{Component Overview}
			
				\begin{itemize}
				
					\item Facial Detection and Tracking (OpenCV, c++)
						\begin{itemize}
							\item Monitor video stream
							\item Identify faces in video frames
							\item Follow faces as they move within a stream
						\end{itemize}
					\item Image Pre-Processor (OpenCV, c++)
						\begin{itemize}
							\item Crop face image so only facial features remain
							\item Align a face to an optimal position
							\item Prepare image (Greyscale, Equalisation etc.)
						\end{itemize}
					\item Facial Feature Extractor (OpenCV, c++)
						\begin{itemize}
							\item Compose Eigenfaces
						\end{itemize}
					\item Facial Feature Matcher (OpenCV, c++)
						\begin{itemize}
							\item Compare Eigenfaces to known faces
							\item If the comparison has a high confidence, a match was found,
							\item Else, a new individual may have been classified
						\end{itemize}
					\item Image Data Manager (OpenCV, c++)
						\begin{itemize}
							\item Manages the image database of known/classified faces
						\end{itemize}
					\item System Administration (Either web based or Java application)
						\begin{itemize}
							\item Provide functionality to effectively and efficiently manage users of the system
							\item Provide functionality to identify and classify faces that are in the known faces database
							\item Provide a monitoring and reporting facility
						\end{itemize}
					\item User Interfaces (Web and Android)
						\begin{itemize}
							\item Provides the interface for users of the system to upload and compare images of persons of interest
						\end{itemize}
					
				\end{itemize}
	\newpage
	
	\section{Project Deliverables}
	
		\subsection{Documentation}
		
			\begin{itemize}
				\item Functional Requirement Specification
				\item Architectural Design Specification
				\item Test Plans
				\item User Manual
				\item Installation Manual
			\end{itemize}
			
		\subsection{Source Code}
		
			\begin{itemize}
				\item All project source code
				\item All test cases
				\item Build scripts
				\item Deployment scripts
			\end{itemize}
			
	
	\section{Communication Channels}
	
		\begin{description}
			\item [Meetings:] BiTwise Developers would like to suggest frequent meetings with the client, at the convenience of the client.
			\item [E-mail:] All BiTwise members will be reachable through their respective e-mail addresses throuhgout the duration of the project.
			\item [Phone:] All BiTwise members will be contactable by their respective cellphone numbers.
			\item [Other:] BiTwise Developers are open to other means of collaboration, for example Google hangouts, or Skype etc.
		\end{description}
\end{document}