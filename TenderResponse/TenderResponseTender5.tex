\documentclass[a4paper]{article}

\usepackage{graphicx}
\usepackage{float}
\usepackage{geometry}

\begin{document}

	\title{Tender Proposal}
	\begin{center}
	\centering 
	\includegraphics[height=10cm]{images}
	\includegraphics[width=10cm]{BiTwise}
				\newline
							\newline
	\end{center}
	\section*
	{
		\begin{center}
			\Huge{Tender Proposal}
			\newline
			\newline
			\Large{CSIR - Mobile Augmented Reality Number Plate Recognition}
		\end{center}
	}
\vspace{0.5cm}
	\section{General}

		\subsection{Purpose of Document}

		This document provides a description of the proposed solution submitted by BiTwise 					Developers in response to the invitation for bids on tender for the Mobile Augmented 				Reality Number Plate Recognition from CSIR.

		\subsection{Introduction}

	The aim of this document is to provide a detailed proposal for Mobile Augmented Reality Number Plate Recognition project that offers users the ability to utilize the smartphone using a mobile live video source to simply do object detection and feature extraction. This application also extends in assisting law enforcement agencies to use this application as a tool to detect motor number plates relating to motor crimes.

Our mission is to provide an application of mobile augmented reality number plate recognition for use in military and police domains in getting additional details of a specific vehicle using a smartphone.


	\section{Executive Summary}

	Bitwise Developers are a unique collection of aspiring IT professionals, each with ambition 		and sound judgement. We are all in our final year of study at the University of Pretoria. 

	The current stakeholders in this project proposal are as follows:
	\begin{itemize}
		\item Company Name : Council of Scientific and Industrial Research (CSIR)
		\item Project Owner : Smart Systems Research Group of the CSIR
		\begin{enumerate}
		\item Priaash Ramadeen - Email:  pramadeen@csir.co.za
		\item Shazia Vawda - Email: svawda@csir.co.za
		\item Pieter Botha - Email : pbotha@csir.co.za
		\end{enumerate}
		\item Team: BiTwise Developers
	\end{itemize}
		
		\subsection{Team Portfolio}
		\begin{itemize}
		\item Team name: BiTwise Developers
        \item Team members: 
        \begin{enumerate}
        \item Priscilla Hammond(Team leader) - Email : u11025477@tuks.co.za
        \item Jacques Lewis - Email : u28183488@tuks.co.za
        \item Francois Oberholzer - Email : u12039803@tuks.co.za
        \end{enumerate}         
		\end{itemize}
		


		\subsection{Unique Differentiators}
		BiTwise Developers is a group formed by 3 students from the Department of Computer 					Science, University of Pretoria. It is a necessary requirement to COS 301: Software 				Engineering, a third year module, that we are to accomplish a project which is industry 			related, which resulted to the formation of this group.

		This group is composed of self-motivated and intellectual persons driven by hard work and 			fortitude to deliver IT solutions of excellence and exceptional quality.

		This group consists of three different persons with three slightly similar degrees in the 			Information Technology field, two BSc IT Software developer students both with apparent 			understanding of system development and the strategies used in software development and a 			BSc IT Psychology (with Artificial Intelligence) student who in apart from knowing 					computer programming will be assisting with how to develop a solution with consideration 			of the users’ psychological needs. All three students are well accustomed to C++, Java, 			Linux and have moderate knowledge in Android. 

		This group will use open-source tools for both the documentation and solution’s 					implementation.


	\section{Technical Proposal}

		\subsection{Project Description}

			\subsubsection{Problem Statement}
			The project request as submitted by Smart Systems Research Group of the CSIR.

An application to complement the real world live view is permitted by Mobile Augmented Reality, with locally processed information and/or information from other bases.  This is all performed in real time. 

This application is for extending the well-known augmented reality technology, which is a location based place search used in locating for landmarks when the smartphone is held up and the application picks up landmarks around and displays them onto the mobile device. This project proposes an extended application of this technology into law enforcement domains, namely for military and police. 

This application’s extension of mobile augmented reality view into real time number plate recognition can be a beneficial tool to benefit mobile units gets supplementary information of a specific automobile using a smartphone. 

Employing that the smartphone do real time detection and feature extraction from a mobile live video source is the project’s core vision. In addition to the function of this application, the information captured on the used smartphone can also then be distributed via web services to all other mobiles units.


		\subsection{Proposed Solution}

		Bitwise Developers believe that we can solve the problem by using our knowledge and 				resources, and by applying our skills as software developers.
		We propose a possible solution as follows:
		
			\subsubsection{Solution Overview}
			\begin{itemize}
			\item Mobile Application : The mobile application will be designed with the following        				functional requirements
			\begin{enumerate}
			\item It should be able to detect South African automobile number plate and their contained text
			\item The user interface should be prove easy and flexible for the users thus making the interface user-friendly
			\item It should post the number plate information to the web service after capturing it on the smartphone
			\item The mobile should be able to detect features through the camera live video
			\item It should receive and extract information to be visualized onto camera view based on the number plate.
			\end{enumerate}
			In the implementation of the mobile application, we will design and construct the Android application using Android SDK and Java combined with Qualcomm Vuforia augmented reality framework. In addition, it should be able to detect the motor description of the scanned number plate.
			\item Web application
			\begin{enumerate}
			\item It should retrieve information from the android application and perform database querying based on the received number plate.
			\item It should post the information in the database back to the android application.
It should display a user friendly client-side application to view all the number plate data saved on the local database.
			\end{enumerate}
			\end{itemize}	


	\section{Implementation Plan}
	This section aims to outline the development of the system of the proposed solution of the project. 
The project team will be submitting the subsequent documents (additional documentation to be provided at a later stage):
\begin{itemize}
\item	An original tender document: This document will provide a comprehensive description of the Mobile Augmented Reality Number Plate Recognition and set out terms under which the tender will be accepted for evaluation. 
\item	A high level requirements document:  This document will provide a detailed description of what the proposed application will do as in, the purpose of the system, system features and interfaces and strictly specifying the system’s functional and non-functional requirements, its quality and data requirements and the limitations under which the system should operate.
\item	An architecture proposal document: This document will provide a comprehensive description
of CSIR Mobile Augmented Reality Number Plate Recognition. This architectural specification will give a detailed view of the
purpose of the system with respect to its overall architecture and architectural features. This will
then formally stipulate the subsystem views, policies, its data requirements, as well as the
limitations under which the system operates.
\item	A user manual: This document will provide all the necessary information giving assistance to specified users on how to use this application for motor number plate recognition.
\item	Test plans: This document will provide the strategy in confirming and certifying that this application met its design specifications and other necessary and specified requirements
\end{itemize}
The project team will also be submitting software deliverables, namely
\begin{itemize}
\item	The designed Android application(.apk file)
\item	The complete source code for both the web and mobile application
\item	Any hardware purchased with CSIR funds, will be delivered to CSIR by the team
\end{itemize}



	\section{Communication Channels}
	\begin{itemize}
	\item	Email: This is the top prioritized way of communicating with clients via email to explain concepts, confirm team’s progress and set up meetings.
\item	Face-to-face: Have meetings at clients’ convenience to explain concepts and confirm team’s progress.
\item	Telephonic: Telephonic consultations to explain concepts and confirm team’s progress.

	\end{itemize}
	
\end{document}